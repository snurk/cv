%%%%%%%%%%%%%%%%%%%%%%%%%%%%%%%%%%%%%%%%%%%%%%%%%%%%%%%%%%%%%%%%%%%%%%%%
%%%%%%%%%%%%%%%%%%%%%% Simple LaTeX CV Template %%%%%%%%%%%%%%%%%%%%%%%%
%%%%%%%%%%%%%%%%%%%%%%%%%%%%%%%%%%%%%%%%%%%%%%%%%%%%%%%%%%%%%%%%%%%%%%%%

%%%%%%%%%%%%%%%%%%%%%%%%%%%%%%%%%%%%%%%%%%%%%%%%%%%%%%%%%%%%%%%%%%%%%%%%
%% NOTE: If you find that it says                                     %%
%%                                                                    %%
%%                           1 of ??                                  %%
%%                                                                    %%
%% at the bottom of your first page, this means that the AUX file     %%
%% was not available when you ran LaTeX on this source. Simply RERUN  %% 
%% LaTeX to get the ``??'' replaced with the number of the last page  %% 
%% of the document. The AUX file will be generated on the first run   %%
%% of LaTeX and used on the second run to fill in all of the          %%
%% references.                                                        %%
%%%%%%%%%%%%%%%%%%%%%%%%%%%%%%%%%%%%%%%%%%%%%%%%%%%%%%%%%%%%%%%%%%%%%%%%

%%%%%%%%%%%%%%%%%%%%%%%%%%%% Document Setup %%%%%%%%%%%%%%%%%%%%%%%%%%%%

% Don't like 10pt? Try 11pt or 12pt
\documentclass[10pt]{article}

% This is a helpful package that puts math inside length specifications
\usepackage{calc}

% Layout: Puts the section titles on left side of page
\reversemarginpar

%
%         PAPER SIZE, PAGE NUMBER, AND DOCUMENT LAYOUT NOTES:
%
% The next \usepackage line changes the layout for CV style section
% headings as marginal notes. It also sets up the paper size as either
% letter or A4. By default, letter was used. If A4 paper is desired,
% comment out the letterpaper lines and uncomment the a4paper lines.
%
% As you can see, the margin widths and section title widths can be
% easily adjusted.
%
% ALSO: Notice that the includefoot option can be commented OUT in order
% to put the PAGE NUMBER *IN* the bottom margin. This will make the
% effective text area larger.
%
% IF YOU WISH TO REMOVE THE ``of LASTPAGE'' next to each page number,
% see the note about the +LP and -LP lines below. Comment out the +LP
% and uncomment the -LP.
%
% IF YOU WISH TO REMOVE PAGE NUMBERS, be sure that the includefoot line
% is uncommented and ALSO uncomment the \pagestyle{empty} a few lines
% below.
%

%% Use these lines for letter-sized paper
\usepackage[paper=letterpaper,
            %includefoot, % Uncomment to put page number above margin
            marginparwidth=1.2in,     % Length of section titles
            marginparsep=.05in,       % Space between titles and text
            margin=1in,               % 1 inch margins
            includemp]{geometry}
\usepackage[utf8]{inputenc}
\usepackage[T2A]{fontenc}

%% Use these lines for A4-sized paper
%\usepackage[paper=a4paper,
%            %includefoot, % Uncomment to put page number above margin
%            marginparwidth=30.5mm,    % Length of section titles
%            marginparsep=1.5mm,       % Space between titles and text
%            margin=25mm,              % 25mm margins
%            includemp]{geometry}

%% More layout: Get rid of indenting throughout entire document
\setlength{\parindent}{0in}

%% This gives us fun enumeration environments. compactenum will be nice.
\usepackage{paralist}

%% Reference the last page in the page number
%
% NOTE: comment the +LP line and uncomment the -LP line to have page
%       numbers without the ``of ##'' last page reference)
%
% NOTE: uncomment the \pagestyle{empty} line to get rid of all page
%       numbers (make sure includefoot is commented out above)
%
\usepackage{fancyhdr,lastpage}
\pagestyle{fancy}
\pagestyle{empty}      % Uncomment this to get rid of page numbers
\fancyhf{}\renewcommand{\headrulewidth}{0pt}
\fancyfootoffset{\marginparsep+\marginparwidth}
\newlength{\footpageshift}
\setlength{\footpageshift}
          {0.5\textwidth+0.5\marginparsep+0.5\marginparwidth-2in}
\lfoot{\hspace{\footpageshift}%
       \parbox{4in}{\, \hfill %
                    \arabic{page} of \protect\pageref*{LastPage} % +LP
%                    \arabic{page}                               % -LP
                    \hfill \,}}

% Finally, give us PDF bookmarks
\usepackage{color,hyperref}
\definecolor{darkblue}{rgb}{0.0,0.0,0.9}
\hypersetup{colorlinks,breaklinks,
            linkcolor=darkblue,urlcolor=darkblue,
            anchorcolor=darkblue,citecolor=darkblue}

%%%%%%%%%%%%%%%%%%%%%%%% End Document Setup %%%%%%%%%%%%%%%%%%%%%%%%%%%%


%%%%%%%%%%%%%%%%%%%%%%%%%%% Helper Commands %%%%%%%%%%%%%%%%%%%%%%%%%%%%

% The title (name) with a horizontal rule under it
%
% Usage: \makeheading{name}
%
% Place at top of document. It should be the first thing.
\newcommand{\makeheading}[1]%
        {\hspace*{-\marginparsep minus \marginparwidth}%
         \begin{minipage}[t]{\textwidth+\marginparwidth+\marginparsep}%
                {\large \bfseries #1}\\[-0.15\baselineskip]%
                 \rule{\columnwidth}{1pt}%
         \end{minipage}}

% The section headings
%
% Usage: \section{section name}
%
% Follow this section IMMEDIATELY with the first line of the section
% text. Do not put whitespace in between. That is, do this:
%
%       \section{My Information}
%       Here is my information.
%
% and NOT this:
%
%       \section{My Information}
%
%       Here is my information.
%
% Otherwise the top of the section header will not line up with the top
% of the section. Of course, using a single comment character (%) on
% empty lines allows for the function of the first example with the
% readability of the second example.
\renewcommand{\section}[2]%
        {\pagebreak[2]\vspace{1.3\baselineskip}%
         \phantomsection\addcontentsline{toc}{section}{#1}%
         \hspace{0in}%
         \marginpar{
         \raggedright \scshape #1}#2}

% An itemize-style list with lots of space between items
\newenvironment{outerlist}[1][\enskip\textbullet]%
        {\begin{enumerate}[#1]}{\end{enumerate}%
         \vspace{-.6\baselineskip}}

% An itemize-style list with little space between items
\newenvironment{innerlist}[1][\enskip\textbullet]%
        {\begin{compactenum}[#1]}{\end{compactenum}}

% To add some paragraph space between lines.
% This also tells LaTeX to preferably break a page on one of these gaps
% if there is a needed pagebreak nearby.
\newcommand{\blankline}{\quad\pagebreak[2]}

\newcommand{\mylink}[2]{\href{#1}{{#2}}}

%%%%%%%%%%%%%%%%%%%%%%%% End Helper Commands %%%%%%%%%%%%%%%%%%%%%%%%%%%

%%%%%%%%%%%%%%%%%%%%%%%%% Begin CV Document %%%%%%%%%%%%%%%%%%%%%%%%%%%%

\begin{document}
\makeheading{Нурк Сергей}

%\section{To Add}
%  Details about extra curricular courses (incl coursera)\\
%  Detailed interests\\
%  Details on dissertation project

\section{Контактная информация}
%  Laboratory of Mathematical Logic\\
%  Petersburg Department of Steklov Institute of Mathematics\\
%  Fontanka 27, St.Petersburg 191023, Russia\\
  Электронная почта: \href{mailto:sergeynurk@gmail.com}{\tt sergeynurk@gmail.com}\\
  Домашняя страница: \href{http://bioinf.spbau.ru/members/sergey-nurk}{\tt 
  http://bioinf.spbau.ru/members/sergey-nurk}\\
  Телефон: +7-950-043-44-06

\section{Место работы}
%\begin{innerlist}
%  \item 
  \href{http://bioinf.spbau.ru}{\tt Проблемная лаборатория вычислительной биологии}\\
  \href{http://spbau.ru}{\tt Санкт-Петербургский академический университет НОЦНТ РАН}\\
  Младший научный сотрудник\\
  февраль 2011--настоящее время
%\end{innerlist}

\section{Биографические данные}
  Дата Рождения: 8 марта 1989\\
  Гражданство: РФ\\
  Семейное положение: Холост

\section{Education}
%
%\href{http://spbau.ru/}{\tt Saint-Petersburg Academic University}\\
%  Ph.D. student, 2011-2014. 
%\begin{innerlist}
%  \item Expected degree: Ph.D. in Computational Biology
%  \item Advisor: \href{http://cseweb.ucsd.edu/~ppevzner/}{\tt Pavel A.~Pevzner}
%  \item Area of study: computational genomics
%  \item Expected defense date: Fall 2015
%\end{innerlist}
%
%\blankline 
\href{http://english.spbu.ru/}{\tt Saint-Petersburg State University}\\
  Ph.D candidate
\begin{innerlist}
  \item Expected degree: PhD in Bioinformatics
  \item Thesis: ``Assembling genomes of non-cultivable microorganisms from high-throughput sequencing data''
  \item Advisor: \href{http://cseweb.ucsd.edu/~ppevzner/}{\tt Pavel A.~Pevzner}
  \item Area of study: \textit{de novo} microbial (single-cell) and metagenomic assembly
  \item Expected defense date: May 2019
\end{innerlist}

\blankline

\href{http://www.math.spbu.ru/en/index.html}{\tt St.~Petersburg State University, Mathematics and Mechanics Faculty, Software Engineering Department 
%specialization : Mathematical support and administration of informational systems
}\\
  M.Sc. in Computer Science, 2006--2011 % Specialist (eq. of Master) in Computer Science.
\begin{innerlist}
  %\item Advisor: Nikolay I.~Vyahhy
  \item Thesis: ``Development of de Bruijn graph processing algorithms
        for genome assembly problem'' (Advisor: \href{http://spbsu.ru/vyahhi/}{\tt Nikolay I.~Vyahhi})
  \item Diploma cum laude
\end{innerlist}

\blankline

Lyceum of Mathematics and Physics \#239, GPA 5.0 (2002--2006)

\section{Additional Education}
%\begin{itemize}
%\item 
\href{http://www.amse.ru/}{Academy of Modern Software Engineering}\\
      Student\\
      2008--2010
%\item 

\blankline

%Computer Science Student Club at PDMI, St.~Petersburg\\
%      Member\\
%      2008--present 

%      Attended extra curriculum courses at the University:
%            Computational Geometry (prof. Kira Vyatkina)
%                    Complexity Theory, quantum computations (prof. Edward Hirsch)
%                            Advanced Algorithms (prof. Andrey Lopatin)
%Clustering (prof. Dmitry Shalimov)

%Coursera :)
%\end{itemize}


\section{Areas of interest}
\begin{innerlist}
%\item bioinformatics
\item genome assembly
\item metagenomics
\item comparative genomics
\item software design
\item data analysis
%\item online courses on statistical learning and data analysis
%\item board games

%\item algorithms design
%\item computational proteomics
%\item machine learning
% data mining
% computational geometry
% complexity theory
%\blankline
\end{innerlist}

\section{Research Papers}
%
%Journals
%Refereed 
%Conference Proceedings
%\begin{enumerate}
\begin{innerlist}
\item 
\textbf{Sergey Nurk}, Pavel A. Pevzner\\
``SPArcle: Identifying Microbial Genome Variations with Colored de Bruijn Graphs''. \\
Manuscript in preparation

\blankline

\item Gabriela I Guzman, Jose Utrilla, \textbf{Sergey Nurk}, Elizabeth Brunk, Jonathan M Monk, Ali Ebrahim, Bernhard Ø Palsson, Adam M. Feist\\
``Model-driven discovery of underground metabolic functions in Escherichia coli''. \\
Accepted to PNAS

\blankline

\item Andrey D. Prjibelski, Irina Vasilinetc, Anton Bankevich, Alexey Gurevich, Tatiana Krivosheeva, \textbf{Sergey Nurk}, Son Pham, Anton Korobeynikov, Alla Lapidus and Pavel A. Pevzner\\
``ExSPAnder: a universal repeat resolver for DNA fragment assembly''. \\
Bioinformatics, 30(12), i293-i301, 2014

\blankline

\item \textbf{Sergey Nurk}$^*$, Anton Bankevich$^*$ ($^*$ equal contribution)%et al.\\
, Dmitry Antipov, Alexey A. Gurevich, Anton Korobeynikov, Alla Lapidus, Andrey D. Prjibelski, Alexey Pyshkin, Alexander Sirotkin, Yakov Sirotkin, Ramunas Stepanauskas, Scott R. Clingenpeel, Tanja Woyke, Jeffrey S. Mclean, Roger Lasken, Glenn Tesler, Max A. Alekseyev, and Pavel A. Pevzner \\
``Assembling Single-Cell Genomes and Mini-Metagenomes From Chimeric MDA Products''. \\
Journal of Computational Biology 20(10), 2013

\blankline

\item \textbf{Sergey Nurk}$^*$, Anton Bankevich$^*$ ($^*$ equal contribution)%et al.\\
, Dmitry Antipov, Alexey A. Gurevich, Anton Korobeynikov, Alla Lapidus, Andrey Prjibelsky, Alexey Pyshkin, Alexander Sirotkin, Yakov Sirotkin, Ramunas Stepanauskas, Jeffrey McLean, Roger Lasken, Scott R. Clingenpeel, Tanja Woyke, Glenn Tesler, Max A. Alekseyev, and Pavel A. Pevzner \\
``Assembling Genomes and Mini-metagenomes from Highly Chimeric Reads''. \\
Proceedings of the 17th Annual International Conference on Research in Computational Molecular Biology, Lecture Notes in Computer Science 7821 (2013).

\blankline

\item Jeffrey S. McLean, Mary-Jane Lombardo, Michael G. Ziegler, Mark Novotny, Joyclyn Yee-Greenbaum, Jonathan H. Badger, Glenn Tesler, \textbf{Sergey Nurk}, Valery Lesin, Daniel Brami, Adam P. Hall, Anna Edlund, Lisa Z. Allen, Scott Durkin, Sharon Reed, Francesca Torriani, Kenneth H. Nealson, Pavel A. Pevzner, Robert Friedman, J. Craig Venter and Roger S. Lasken \\
``Genome of the pathogen Porphyromonas gingivalis recovered from a biofilm in a hospital sink using a high-throughput single cell genomic platform''. \\
Genome research, 23(5), 867-877

\blankline

\item Anton Bankevich$^*$, \textbf{Sergey Nurk}$^*$ ($^*$ equal contribution)%et al.\\
, Dmitry Antipov, Alexey Gurevich, Mikhail Dvorkin, Alexander Kulikov, Valery Lesin, Sergey Nikolenko, Son Pham, Andrey Prjibelski, Alexey Pyshkin, Alexander Sirotkin, Nikolay Vyahhi, Glenn Tesler, Max Alekseyev and Pavel Pevzner \\
``SPAdes: a New Genome Assembler and its Applications to Single Cell Sequencing''.\\
Journal of Computational Biology 19(5), 2012
%\end{innerlist}
%\end{enumerate}

\blankline

%Miscellaneous
\item \textbf{Sergey Nurk} \\
``An $O(2^{0.4058m})$ Upper Bound for \textit{Circuit SAT}.''\\
PDMI preprint 10/2009, 2009
\end{innerlist}
%\end{enumerate}


\section{Attended Student Schools and Workshops}
%
\begin{innerlist}

\item Systems Biology Workshop. \\
Saint-Petersburg, Russia, 2015

\item Systems Biology Workshop. \\
Saint-Petersburg, Russia, 2014

\item Microsoft School on Algorithms for Massive Data (ALMADA). \\
Moscow, Russia, 2013

% \item $17^{th}$ Conference on Research in Computational Molecular Biology.\\ 
% Beijing, China, 2013.
% 
% \item $16^{th}$ Conference on Research in Computational Molecular Biology.\\ 
% Barcelona, Spain, 2012.
% 
% \item $10^{th}$ European Conference on Computational Biology.\\ 
% Vienna, Austria, 2011.

\item Russian Summer School in Information Retrieval (RuSSIR).\\ 
Voronezh, Russia, 2010

\item Microsoft Data Structures and Algorithms School (MIDAS).\\ 
Saint-Petersburg, Russia, 2010

\item NoNA Summer School on Complexity Theory.\\
Saint-Petersburg, Russia, 2009

\item Joint Advanced Student School (JASS).\\
Saint-Petersburg, Russia, 2009.\\ 
Topic: Propositional Proof Complexity

\end{innerlist}

\section{Доклады и постеры}
\begin{innerlist}
 \item Доклад ``Графы де Брюина и алгоритмы сборки геномов''\\
       Летняя биоинформатическая школа\\
       Москва, Россия, 2013
 \item Доклад ``Assembling Genomes and Mini-metagenomes from Highly Chimeric Reads''
       $17^{th}$ Conference on Research in Computational Molecular Biology.\\ 
       Beijing, China, 2013.
 \item Постер ``Expandable de novo genome assembler for short-read sequence data.''\\
       ISMB/ECCB, совместная работа с Н.Вяххи, А.Банкевичем, М.Алексеевым и П.певзнером.\\ 
       Vienna, Austria, 2011
 \item Доклад об инструменте для сбора информации с веб страниц, разработанного в моем отделе в компании Яндекс\\
       Yet Another Conference (YaC'2010)\\
       Москва, Россия, 2010
 \item Доклад ``Upper bound for Circuit SAT''\\
       Estonian Theory Days 2009\\
       Palmse, Estonia, 2009
 \item Доклад ``Lower bounds for $k-DNF$ resolution on random $3-CNF$s''\\
       Joint Advanced Student School (JASS'2009)\\
       Санкт-петербург, Россия, 2009
\end{innerlist}

%%\newcommand{\ignoregrant}[1]{}

\section{Grants}
%\begin{innerlist}
Participant of a grant of the Federal Target Programme “Scientific and
scientific-pedagogical personnel of the innovative Russia” (contract \#P265
from 23.07.2009)
%\end{innerlist}



\section{Industrial Experience}
\href{http://company.yandex.com/}{\tt Yandex company} (Yandex is the largest search engine in Russia and develops a number of Internet-based services and products)\\
Structured Web Mining Department
\begin{innerlist}
\item Software Engineer (Java)\\
  2009--2011
\item Software Engineering intern\\
  Summer 2009
\end{innerlist}


\section{Teaching Experience}
``Comparative genomics'' student seminar at St. Petersburg Academic University\\
Supervisor\\
Spring $2014$

\blankline

``Molecular sequence analysis'' course at St. Petersburg Academic University\\
Lecturer\\
Spring $2014$

\blankline

Java programming language course at St. Petersburg Academic University\\
Teaching assistant\\ 
$2012-2013$




\section{Relevant Skills}
%Programming languages
\begin{innerlist}
  \item Java
  \item C++
  \item Python
  \item R
%  \item C$\#$
  \item SQL, XML
  \item \LaTeX, bash, vim, git
  %\item Python
\end{innerlist}

\section{Languages}
\begin{innerlist}
  \item Russian: native 
  \item English: fluent
\end{innerlist}

%\section{Interests and dissertation project description}
\blankline
In general, my primary areas of interest are algorithms design and software design. 
\blankline
I'm really fond of working on algorithms for processing large amounts of data.
Recently I have been developing algorithms that should effectively process graphs with tens of millions of vertices that arise in the problem of genome assembly.
\blankline
Also some time ago I've become very interested in machine learning and I study various related topics in my free time.

\blankline
Since I've become junior research fellow at the Algorithmic Biology Lab at Saint-Petersburg Academic University, my primary research interests lie in the field of computational genomics. 

My dissertation project consists of two parts. 

\blankline
First part concerns the design of genomic assembler for the data originated from ``single-cell'' DNA sequencing technologies. 
In the last few years, analysis of single-cells became one of the hottest topics in genomics.
To put it simple, de novo genome assembly problem is the task of reconstruction long sequences of DNA from its short overlapping fragments. 
There are many good software tools to perform genome assembly, but none of them is able to produce satisfactory results for the data, obtained in the single-cell experiments. 
I'm working on the algorithmic approaches to overcome the two major difficulties in assembly of single-cell: 
highly non-uniform coverage of DNA sequence with short fragments and highly elevated level of chimerism (chimera is a fragment that  of two parts obtained from distant parts of DNA sequence).

\blankline
Second part of my dissertation project is development of a novel framework for simultaneous comparative analysis of multiple genomes and differential assembly.
Simultaneous comparison of multiple genomes is very important for many biological applications, such as evolutionary studies and analysis of bacterial pathogenicity factors. 
Unfortunately, state of the art computational approaches have multiple serious limitations that make them hard or impossible to apply in many contemporary biological studies.
For example, they can only work with fully (or almost fully) reconstructed genomes, but in most present-day cases only draft assemblies are available.
I'm working on an approach that seems promising in overcoming some of these limitations.
Also this approach should allow to successfully perform differential assembly: assembly of sequencing data in case when closely related species genome has already been fully reconstructed.


\end{document}
