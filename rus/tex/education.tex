\section{Образование}
%
\href{http://spbau.ru/}{\tt Санкт-Петербургский Академический университет НОЦНТ РАН}\\
  Аспирант с 2011 года. Ожидаемая степень: кандидат физ-мат. наук
\begin{innerlist}
  \item Научный руководитель: \href{http://cseweb.ucsd.edu/~ppevzner/}{\tt П.А. Певзнер}
  \item Тематика работы: вычислительная геномика
\end{innerlist}

\blankline 

\href{http://www.math.spbu.ru/en/index.html}{\tt Санкт-Петербургский государственный университет, Математико-механический факультет, спец. Мат. обеспечение и администрирование информационных систем 
%specialization : Mathematical support and administration of informational systems
}\\
  Специалист, 2006--2011 % Specialist (eq. of Master) in Computer Science.
\begin{innerlist}
  %\item Advisor: Nikolay I.~Vyahhy
  \item Тема диплома: ``Разработка алгоритмов обработки графа де Брюина 
  в задаче геномного ассемблирования'' (Руководитель: \href{http://spbsu.ru/vyahhi/}{\tt Н.И. Вяххи})
  \item Диплом с отличием
\end{innerlist}

\blankline

Физико-математический лицей \#239, средний балл 5.0 (2002--2006)

\section{Дополнительное образование}
%\begin{itemize}
%\item 
\href{http://www.amse.ru/}{Академия современного программирования}\\
      Студент\\
      2008--2010
%\item 

\blankline

%Computer Science Student Club at PDMI, St.~Petersburg\\
%      Member\\
%      2008--present 

%      Attended extra curriculum courses at the University:
%            Computational Geometry (prof. Kira Vyatkina)
%                    Complexity Theory, quantum computations (prof. Edward Hirsch)
%                            Advanced Algorithms (prof. Andrey Lopatin)
%Clustering (prof. Dmitry Shalimov)

%Coursera :)
%\end{itemize}

