\section{Interests and dissertation project description}
\blankline
In general, my primary areas of interest are algorithms design and software design. 
\blankline
I'm really fond of working on algorithms for processing large amounts of data.
Recently I have been developing algorithms that should effectively process graphs with tens of millions of vertices that arise in the problem of genome assembly.
\blankline
Also some time ago I've become very interested in machine learning and I study various related topics in my free time.

\blankline
Since I've become junior research fellow at the Algorithmic Biology Lab at Saint-Petersburg Academic University, my primary research interests lie in the field of computational genomics. 

My dissertation project consists of two parts. 

\blankline
First part concerns the design of genomic assembler for the data originated from ``single-cell'' DNA sequencing technologies. 
In the last few years, analysis of single-cells became one of the hottest topics in genomics.
To put it simple, de novo genome assembly problem is the task of reconstruction long sequences of DNA from its short overlapping fragments. 
There are many good software tools to perform genome assembly, but none of them is able to produce satisfactory results for the data, obtained in the single-cell experiments. 
I'm working on the algorithmic approaches to overcome the two major difficulties in assembly of single-cell: 
highly non-uniform coverage of DNA sequence with short fragments and highly elevated level of chimerism (chimera is a fragment that  of two parts obtained from distant parts of DNA sequence).

\blankline
Second part of my dissertation project is development of a novel framework for simultaneous comparative analysis of multiple genomes and differential assembly.
Simultaneous comparison of multiple genomes is very important for many biological applications, such as evolutionary studies and analysis of bacterial pathogenicity factors. 
Unfortunately, state of the art computational approaches have multiple serious limitations that make them hard or impossible to apply in many contemporary biological studies.
For example, they can only work with fully (or almost fully) reconstructed genomes, but in most present-day cases only draft assemblies are available.
I'm working on an approach that seems promising in overcoming some of these limitations.
Also this approach should allow to successfully perform differential assembly: assembly of sequencing data in case when closely related species genome has already been fully reconstructed.
